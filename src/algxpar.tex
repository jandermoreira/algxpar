%! Author = Jander Moreira
%! Date = 08/05/2023
\documentclass[12pt]{ltxdoc}
\usepackage[T1]{fontenc}

\usepackage{algxpar}
\usepackage[showframe]{geometry}
\geometry{top = 2.5cm, bottom = 2cm, right = 2.5cm, left = 3cm}
\usepackage[backref=page]{hyperref}
\hypersetup{
    colorlinks,
    urlcolor = blue!20!black,
    linkcolor = blue!10!black,
    citecolor = black!80,
}
\usepackage{imakeidx}
\makeindex
\usepackage{listings}
\usepackage[outputdir=../out]{minted}
\usepackage{tcolorbox}
\tcbuselibrary{listings, minted}
\tcbset{listing engine = minted}
\usepackage{tikz}


\newenvironment{commands}{
    \VerbatimEnvironment
    \tcolorbox[
    bottomrule = 0pt,
    toprule = 0pt,
    leftrule = 0pt,
    rightrule = 0pt,
    titlerule = 0pt,
    colframe = white,
    colback = white,
    sharp corners,
    pad at break*=1pc,
    enhanced jigsaw,
    overlay first and middle={
        \coordinate (A1) at ($(interior.south east) + (-10pt,5pt)$);
        \coordinate (C1) at ($(interior.south east) + (-6pt,7.5pt)$);
        \draw[fill=tcbcolframe] (A1) -- +(0,5pt) -- (C1) -- cycle;
    },
    ]%
    \Verbatim[gobble = 4, tabsize=4, commandchars = &\[\]]%
    }
%     {%
%     \endVerbatim%
%     \endtcolorbox%
% }

        \newtcolorbox{codigo}[1]{%
    bottomrule = 0pt,
    toprule = 0pt,
    leftrule = 0pt,
    rightrule = 0pt,
    titlerule = 0pt,
    sharp corners,
    sidebyside,
    sidebyside align = top,
    #1
}

\usetikzlibrary{calc}
\newenvironment{example}{
    \begingroup\tcbverbatimwrite{\jobname_code.tmp}
    }{
    \endtcbverbatimwrite\endgroup%
    \begin{tcolorbox}
        [
        bottomrule = 0pt,
        toprule = 0pt,
        leftrule = 0pt,
        rightrule = 0pt,
        titlerule = 0pt,
        colframe = white,
        segmentation style = {black!30, solid},
        colback = white,
        sharp corners,
        pad at break* = 1pc,
        enhanced jigsaw,
        breakable,
        overlay first and middle={
            \coordinate (A1) at ($(interior.south east) + (-10pt,5pt)$);
            \coordinate (C1) at ($(interior.south east) + (-6pt,7.5pt)$);
            \draw[fill=tcbcolframe] (A1) -- +(0,5pt) -- (C1) -- cycle;
        },
        ]%
        \small\VerbatimInput[tabsize=4, gobble = 2]{\jobname_code.tmp}
        \tcblower
% 	\begin{minipage}{0.85\linewidth}
        \begin{algorithmic}
            \input{\jobname_code.tmp}
        \end{algorithmic}
% 	\end{minipage}
    \end{tcolorbox}
}


% ShowMacro: showw macro name and description
\NewDocumentCommand{\ShowMacro}{ m m }{%
    \begin{itemize}
        \item[\textbackslash\texttt{#1}] #2
    \end{itemize}
}


\title{The \textsf{algxpar} package\thanks{This document corresponds to \textsf{algxpar}~\AlgVersion, dated \AlgDate.}}
\author{Jander Moreira\\\texttt{moreira.jander@gmail.com}}
\date{\today}

\begin{document}
\maketitle

\begin{abstract}
    The \textsf{algxpar} packages is an extension of the \textsf{algorithmicx} package to handle multiline text with the proper indentation.
\end{abstract}

\tableofcontents
\vspace{2em}

% \changes{v0.9}{2019/10/24}{First version}
% \changes{v0.91}{2020/04/30}{Macro now can be used as super-/subscripts in math formulas, while still preventing hyphenaton in text mode.}
% \changes{v0.91}{2020/06/14}{New macro for assignments, using $\gets$}
% \changes{v0.91}{2020/06/14}{New macro for assignments (verbose)}


\section{Introduction}
I teach algorithms and programming and have adopted the \textsf{algorithmicx} package (\textsf{algpseudocode}) to write my algorithms as it provides clear and easy-to-read pseudocodes with a minimum of effort to achieve visually pleasing code.

As part of the teaching process, I use very detailed commands in my algorithms before students start using a more synthetic form. For example, I initially write ``Start a counter $c$ with the value $0$'', which later becomes ``${c \gets 0}$''. This leads to sentences that often span text over two or more lines, especially in two-column documents with nested commands.

Unfortunately, \textsf{algorithmx} doesn't natively support multiline statements. This package therefore extends the macros to handle multiple lines properly. Some new commands and features were also added.


\section{Package usage and options}
To use the package simply add it to preamble.

\begin{verbatim}
\usepackage{algxpar}
\end{verbatim}

Package options:
\begin{itemize}
    \item[\texttt{noend}] Suppress the line at the end of a block.
    \item[\texttt{english}] Selects english keywords (default).
    \item[\texttt{brazilian}] Selects portuguese/brazilian keywords.
\end{itemize}


\section{Writting pseudocode}
All pseudocode is written inside the \texttt{algorithmic} environment.

\subsection{Documentation}
A series of macros are defined to provide the header documentation for a pseudocode. The basice are:
\ShowMacro{Description}{General description of the algorithm.}
\ShowMacro{Require}{Preconditions.}
\ShowMacro{Ensure}{Postconditions.}

\begin{tcblisting}{}
    \begin{algorithmic}
        \Description Calculation of the factorial of a natural number through successive multiplications
        \Require $n \in \mathbb{N}$
        \Ensure $f = n!$
    \end{algorithmic}
\end{tcblisting}

Also are provided:
\ShowMacro{Input}{Inputs.}
\ShowMacro{Output}{Outputs}

\begin{tcblisting}{}
    \begin{algorithmic}
        \Description Calculation of the factorial of a natural number through successive multiplications
        \Input $n$ (integer)
        \Output $n!$ (integer)
    \end{algorithmic}
\end{tcblisting}

\subsection{Constants and Identifiers}

\subsection{Assignments and I/O}

\subsection{Comments}

\subsection{Statements}

\subsection{Flow Control}

\subsection{Procedures and Functions}


\section{Customization and Fine Tunning}


\end{document}