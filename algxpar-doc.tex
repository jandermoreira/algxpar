%! Author = Jander Moreira
%! Email =  moreira.jander@gmail.com

\documentclass[a4paper, 11pt]{article}
\usepackage[T1]{fontenc}

\usepackage{amsfonts}
\usepackage{amsmath}
\usepackage{textcomp}
\usepackage[all]{nowidow}

% \usepackage{array}

\usepackage{enumitem}
\setlist{nosep}

\usepackage[
    brazilian,
    language = english,
]{algxpar}


%% Layout

% geometry
\usepackage{geometry}
\geometry{top = 2.5cm, bottom = 2cm, right = 2.5cm, left = 4cm}

% hyperref
\usepackage{hyperref}
\hypersetup{
    colorlinks,
    urlcolor = blue!20!black,
    linkcolor = blue!10!black,
    citecolor = black!80,
}

% cleveref
\usepackage{cleveref}

% makeidx
\usepackage{makeidx}
\makeindex

% minted
\usepackage[outputdir = ./out]{minted}
% \usemintedstyle{borland}
\newminted{latex}{autogobble, breaklines, bgcolor = blue!5, fontsize = \footnotesize}
\newmintinline{latex}{}

% tcolorbox
\usepackage{tcolorbox}
\usepackage{color}
\usepackage{comment}
\tcbuselibrary{skins, listings, minted, breakable}
\tcbset{
    colback = blue!3,
    sharp corners,
    box align = top,
    boxrule = 0pt,
    fontupper = \footnotesize,
    fontlower = \footnotesize,
    minted options={
        fontsize = \footnotesize,
        breaklines,
        autogobble,
    },
    listing engine = minted,
}

% Versions
\usepackage{versionchanges}

%% Text support

% macro arguments formats
\colorlet{argumentcolor}{orange!50!black}
\NewDocumentCommand{\Argument}{ m }{%
    \textcolor{argumentcolor}{$\langle$\normalfont\small\textsl{#1}$\rangle$}%
}
\NewDocumentCommand{\MArg}{ m }{\mbox{\texttt{\{}\Argument{#1}\texttt{\}}}}
\NewDocumentCommand{\OArg}{ m }{\mbox{\texttt{[}\Argument{#1}\texttt{]}}}
\NewDocumentCommand{\LArg}{ m }{\mbox{\texttt{<}\Argument{#1}\texttt{>}}}
\NewDocumentCommand{\PackageName}{ m }{\mbox{\textsf{#1}}}
\NewDocumentCommand{\Deprecated}{}{\textcolor{red!80!black}{(deprecated)}}
\NewDocumentCommand{\FromPackage}{ m }{%
    \tikz\node[draw, rounded corners = 1.5pt, inner sep = 1.5pt,
        font = \sffamily\tiny] {#1};%
}
\NewDocumentCommand{\Empty}{}{%
    \mbox{\normalfont\textcolor{black!60}{\textsl{--empty--}}}
}
\NewDocumentCommand{\Option}{ m }{%
    \mbox{\textcolor{green!40!black}{\texttt{#1}}}%
}
\NewDocumentCommand{\OptionInd}{ m }{%
    \index{#1@\texttt{#1}}%
    \Option{#1}%
}
\NewDocumentCommand{\OptionRef}{ m }{%
    \hyperref[option:#1]{\Option{#1}}%
}
\NewDocumentCommand{\Macro}{ m }{%
    \expandafter\latexinline\expandafter{\csname#1\endcsname}%
}
\NewDocumentCommand{\MacroRef}{ m }{%
    \hyperref[macro:#1]{\Macro{#1}}%
}
\NewDocumentCommand{\MacroDef}{ m }{%
    \index{#1@\texttt{\textbackslash #1}}%
    \refstepcounter{MacroCounter}%
    \label{macro:#1}%
    \Macro{#1}%
}
\NewDocumentCommand{\MacroRefInd}{ m }{%
    \index{#1@\texttt{\textbackslash #1}}%
    \MacroRef{#1}%
}

\tcbset{
    description/.style = {
        coltitle = black,
        fontupper = \normalsize,
        colbacktitle = white,
        titlerule = 0.001pt,
        enhanced jigsaw,
        breakable,
        width = \dimexpr \linewidth - 2em \relax,
        flush right,
        top = 0.5ex,
        bottom = 0pt,
        left = 0pt,
        right = 0pt,
        opacitybacktitle = 0,
        opacityframe = 0,
        opacityback = 0,
    }
}

\NewDocumentEnvironment{macro}{ m O{} o }{%
    %! formatter = off
    \index{#1@\texttt{\textbackslash#1}}%
    \refstepcounter{MacroCounter}%
    \label{macro:#1}%
    %! parser = off
    \IfValueTF{#3}{%
        \begin{macro*}{#1}{#2}
    }{%
        \begin{macro*}{#1}{#2}[#3]%
    }
    %! parser = on
        }{%
    %! parser = off
    \end{macro*}
    %! parser = on
    %! formatter = on
}
\newcounter{MacroCounter}
\NewDocumentEnvironment{macro*}{ m m o }{
    \medskip\par%
    \begin{tcolorbox}[
        title = {\hspace{-2em}\Macro{#1}#2\IfValueT{#3}{\latexinline!{#3} !}},
        description,
    ]
    }{
    \end{tcolorbox}%
    \medskip%
}

\newlength{\docassignment}
\NewDocumentEnvironment{option}{ m m o }{%
    \label{option:#1}%
    \settowidth{\docassignment}{#2}%
    \begin{tcolorbox}
        [
        title = {%
            \hspace{-2em}\OptionInd{#1}%
            \ifdim\docassignment>0pt\Option{ = #2}\fi%
            \IfValueT{#3}{\hfill\textit{Default:} \Option{#3}}
        },
        description,
        ]
            }{
    \end{tcolorbox}%
    \medskip%
}
\NewDocumentEnvironment{option*}{ m }{%
    \begin{tcolorbox}[title = {\hspace{-2em}#1}, description]
    }{
    \end{tcolorbox}%
    \medskip%
}
\NewDocumentEnvironment{optionnoind}{ m m }{%
    \begin{tcolorbox}[
        title = {\hspace{-2em}\Option{#1 = #2}},
        description,
    ]
    }{
    \end{tcolorbox}%
    \medskip%
}

\begin{filecontents}[overwrite, nosearch, noheader]{algxpar-lzw.tex}
    \begin{algorithmic}[1]
        \Description LZW Compression using a table with all known sequences of bytes.
        \Input A flow of bytes
        \Output A flow of bits with the compressed representation of the input bytes
        \Statex
        \Statep{Initialize a table with all bytes}[each position of the table has a single byte]
        \Statep{Initilize \Id{sequence} with the first byte in the input flow}
        \While{there are bytes in the input}[wait until all bytes are processed]
            \Statep{Get a single byte from input and store it in \Id{byte}}
            \If{the concatention of \Id{sequence} and \Id{byte} is in the table}
                \Statep{Set \Id{sequence} to $\Id{sequence} + \Id{byte}$}[concatenate without producing any output]
            \Else
                \Statep{Output the code for \Id{sequence}}[i.e., the binary representation of its position in the table]\label{alg:lzw:output}
                \Statep{Add the concatention of \Id{sequence} and \Id{byte} to the table}[the table learns a longer sequence]\label{alg:lzw:add-to-table}
                \Statep{Set \Id{sequence} to \Id{byte}}[starts a new sequence with the remaining byte]
            \EndIf
        \EndWhile
        \Statep{Output the code for \Id{sequence}}[the remaining sequence of bits]
    \end{algorithmic}
\end{filecontents}

%% Repetitive text
\NewDocumentCommand{\MacroOptionsText}{}{%
    Any \Argument{options} specified uniquely affect this macro.%
}
\NewDocumentCommand{\BlockOptionsText}{}{%
    Any of the \Argument{options} specified in this macro will affect this command and all items in the inner block, propagating up to and including the closing macro.%
}

%%%%%%%%%%%%%%%%%%%%%%%%%%%%%%%%%%%%%%%%%%%%%%%%%%%%%%%%%%%%%%%%%%%%%%

\title{%
    The \PackageName{algxpar} package\thanks{This document corresponds to \PackageName{algxpar}~v\AlgVersion, dated \AlgDate.
    This text was last revised \today.}%
}
\author{Jander Moreira -- \texttt{moreira.jander@gmail.com}}
\date{June 26, 2023}

%%%%%%%%%%%%%%%%%%%%%%%%%%%%%%%%%%%%%%%%%%%%%%%%%%%%%%%%%%%%%%%%%%%%%%


\begin{document}
\maketitle
\sloppy

\begin{abstract}
    The \PackageName{algxpar} package is an extension of the \PackageName{algorithmicx}\footnote{\url{https://ctan.org/pkg/algorithmicx}.}/\PackageName{algpseudocode} package to handle multi-line text with proper indentation and provide a number of other improvements.
\end{abstract}

\tableofcontents

\VCRegisterVersion{0.9}{2019-11-12}
\VCRegisterVersion{0.91}{2020-08-01}
\VCRegisterVersion{0.99}{2023-06-28}
\VCRegisterVersion{0.99.1}{2024/05/05}

\begin{figure}
    \tcbinputlisting{listing file = algxpar-lzw.tex, listing only}
    \vspace{3em}
    \tcbinputlisting{listing file = algxpar-lzw.tex, text only}
\end{figure}


% \vspace{2em}

% \changes{v0.9}{2019/10/24}{First version}
% \changes{v0.91}{2020/04/30}{Macro now can be used as super-/subscripts in math formulas, while still preventing hyphenaton in text mode.}
% \changes{v0.91}{2020/06/14}{New macro for assignments, using $\gets$}
% \changes{v0.91}{2020/06/14}{New macro for assignments (verbose)}

\VCPrintChanges


\section{Introduction}
I teach algorithms and programming and have adopted the \PackageName{algorithmicx} package (\PackageName{algpseudocode}) for writing my algorithms as it provides clear and easy to read pseudocodes with minimal effort to get a visually pleasing code.

The process of teaching algorithms requires a slightly different use of pseudocode than that normally presented in scientific articles, in which the solutions are presented in a more formal and synthetic way. Students work on more abstract algorithms often preceding the actual knowledge of a programming language, and thus the logic of the solution is more relevant than the variables themselves. Likewise, the use of the development strategy by successive refinements also requires a less programmatic and more verbose code. Thus, when discussing the reasoning for solving a problem, it is common to use sentences such as ``\emph{accumulate current expenses in the total sum of costs}'', because ``${s \gets s + c}$'' is, in this case, too synthetic and necessarily involves knowing how variables work in programs.

The consequence of more verbose pseudocode leads, however, to longer sentences that often span two or more lines. As pseudocodes, by nature, value visual organization, with regard to control structures and indentations, it became necessary to develop a package that supports the use of commands and comments that could be easily displayed when more than one line was needed.

The \PackageName{algorithmx} and \PackageName{algpseudocode} packages do not natively support multi-line statements. This package therefore extends several macros to handle multiple lines correctly. Some new commands and a number of features have also been added.

This package was first realeased as V0.9%
\VCChange[disable]{
    type = released,
    version = 0.9,
    description = Initial version,
}%
and almost totally rewritten in v0.99%
\VCChange[disable]{
    type = new,
    version = 0.99,
    description = The code was reviewd and almost entirely rewritten,
}.


\section{Package usage and options}\label{sec:package-usage-and-options}
This package depends on the following packages:

\begin{center}
    \begin{tabular}{ll}
        \PackageName{algorithmicx}  & (\url{https://ctan.org/pkg/algorithmicx})  \\
        \PackageName{algpseudocode} & (\url{https://ctan.org/pkg/algorithmicx})  \\
        \PackageName{amssymb}       & (\url{https://ctan.org/pkg/amsfonts})      \\
        \PackageName{fancyvrb}      & (\url{https://ctan.org/pkg/fancyvrb})      \\
        \PackageName{pgfmath}       & (\url{https://ctan.org/pkg/pgf})           \\
        \PackageName{pgfopts}       & (\url{https://ctan.org/pkg/pgf})           \\
        \PackageName{ragged2e}      & (\url{https://ctan.org/pkg/ragged2e})      \\
        % \PackageName{tcolorbox}     & (\url{https://www.ctan.org/pkg/tcolorbox}) \\
        \PackageName{varwidth}      & (\url{https://www.ctan.org/pkg/varwidth})  \\
        \PackageName{xcolor}        & (\url{https://www.ctan.org/pkg/xcolor})    \\
    \end{tabular}
\end{center}

\medskip
To use the package, simply request its use in the preamble of the document.

\begin{macro*}{usepackage}{\OArg{package options list}}[algxpar]
    Currently, the list of package options includes the following.

    \begin{option*}{\Argument{language name}}
        By default, algorithm keywords are developed in English. The English language keyword set is always loaded. When available, other sets of keywords in other languages can be used simply by specifying the language names. The last language in the list is automatically set as the document's default language.

        Currently supported languages:
        \begin{itemize}
            \item \OptionInd{english} (default language, always loaded)
            \item \OptionInd{brazilian} Brazilian Portuguese
        \end{itemize}
    \end{option*}

    \begin{latexcode}
        % Loads Brazilian keyword set and sets it as default
        \usepackage[brazilian]{algxpar}
    \end{latexcode}

    \begin{optionnoind}{language}{\Argument{language name}}
        This option chooses the set of keywords corresponding to \Argument{language name} as the default for the document. This option is available as a general option (see \OptionRef{language}).

        This option is useful when other languages are loaded.
    \end{optionnoind}

    \begin{latexcode}
        % Loads Brazilian keyword set but keeps English as default
        \usepackage[brazilian, language = english]{algxpar}
    \end{latexcode}

    \begin{option*}{\Option{noend}}
        The \OptionInd{noend} suppresses the line that indicates the end of a block, keeping the indentation.

        See more information in \OptionRef{end} and \OptionRef{noend} options.
    \end{option*}

    \begin{latexcode}
        % Suppresses all end-lines that close a block
        \usepackage[noend]{algxpar}
    \end{latexcode}
\end{macro*}


\section{Writting pseudocode}
Algorithms, following the functionality of the \PackageName{algorithmicx} package, are written within the \latexinline{algorithmic} environment. The possibility of using a number to determine how the lines will be numbered is maintained as in the original version.

An algorithm is composed of instructions and control structures such as conditionals and loops. And also, some documentation and comments.

\begin{tcblisting}{}
    \begin{algorithmic}
        \Description Calculation of the factorial of a natural number
        \Input $n \in \mathbb{N}$
        \Output $n!$
        \Statex
        \Statep{\Read $n$}
        \Statep{$\Id{factorial} \gets 1$}[$0! = 1! = 1$]
        \For{$k \gets 2$ \To $n$}[from 2 up]
            \Statep{$\Id{factorial} \gets \Id{factorial} \times k$}[$(k-1)! \times k$]
        \EndFor
        \Statep{\Write \Id{factorial}}
    \end{algorithmic}
\end{tcblisting}

\subsection{A preamble on comments}\label{sec:a-preamble-on-comments}
This is the Euclid's algorithm as provided in the \PackageName{algorithmicx} package documentation\footnote{A label was suppressed here.}.

\begingroup
%! formatter = off
\AlgSet{comment font = {}}
\begin{tcblisting}{}
    \begin{algorithmic}[1]
        \Procedure{Euclid}{$a,b$}
            \Comment{The g.c.d. of a and b}
            \State $r\gets a\bmod b$
            \While{$r\not=0$}\Comment{We have the answer if r is 0}
                \State $a\gets b$
                \State $b\gets r$
                \State $r\gets a\bmod b$
            \EndWhile
            \State \textbf{return} $b$\Comment{The gcd is b}
        \EndProcedure
    \end{algorithmic}
\end{tcblisting}
%! formatter = on
\endgroup

Comments are added \textit{in loco} with the \MacroRef{Comment} macro, which makes them appear along the right margin. The \PackageName{algxpar} package embeded comments as part of the commands themselves in order to add multi-line support.

Until \PackageName{algxpar} v0.95, they could be added as an optional parameter before the text, in the style of most \LaTeX\ macros.

%! formatter = off
\begingroup
\AlgSet{comment font = {}}
\begin{tcblisting}{}
    \begin{algorithmic}[1]
        \Procedure[The g.c.d. of a and b]{Euclid}{$a,b$}  % <-- Comment
            \State $r\gets a\bmod b$
            \While[We have the answer if r is 0]{$r\not=0$}  % <-- Comment
                \State $a\gets b$
                \State $b\gets r$
                \State $r\gets a\bmod b$
            \EndWhile
            \Statep[The gcd is b]{\Keyword{return} $b$}   % <-- Comment
        \EndProcedure
    \end{algorithmic}
\end{tcblisting}
\endgroup
%! formatter = on

Using the comment before the text always bothered me somewhat, as it seemed more natural to put it after.
\VCChange{
    type = new,
    version = 0.99,
    description = Optional comments may be specified after the command
}%
Thus, as of v0.99, the comment can be placed after the text (as the second parameter of the macro), certainly making writing algorithms more user-friendly. To maintain backward compatibility, the use of comments before text is still supported, although it is discouraged.

\VCChange{
    type = new,
    version = 0.99,
    description = Lines that closes blocks can have comments
}%
In addition to this change, the use of comments in the new format has been extended to most pseudocode macros, such as \MacroRef{EndWhile} for example.

%! formatter = off
\begingroup
\AlgSet{comment font = {}}
\begin{tcblisting}{}
    \begin{algorithmic}[1]
        \Procedure{Euclid}{$a,b$}[The g.c.d. of a and b]  % <-- Comment
            \State $r\gets a\bmod b$
            \While{$r\not=0$}[We have the answer if r is 0]  % <-- Comment
                \State $a\gets b$
                \State $b\gets r$
                \State $r\gets a\bmod b$
            \EndWhile[end of the loop]  % <-- Comment
            \Statep{\Keyword{return} $b$}[The gcd is b]  % <-- Comment
        \EndProcedure
    \end{algorithmic}
\end{tcblisting}
\endgroup
%! formatter = on

Using \MacroRef{Comment} still produces the expected result, although it may break automatic tracking of longer lines.

Throughout this documentation, former style comments are denoted as \Argument{comment*}, while the new format uses \Argument{comment}.

See more about comments in \cref{sec:comments}.

\subsection{A preamble on options}\label{sec:a-preamble-on-options}
\VCChange{
    type = new,
    version = 0.99,
    description = {The ``look and feel'' of the pseudocode can be set in the \Macro{begin}\latexinline{{algorithmic}}}
}[Preamble options]%
As of version 0.99, a list of options can be added to each command, changing some algorithm presentation settings. These settings are optional and must be entered using angle brackets at the end of the command.

%! formatter = off
\begin{tcblisting}{}
    \begin{algorithmic}<keyword font = \scshape\bfseries, comment width = nice>
        \If{$a > b$}[check conditions]
            \While{$a > 0$}<keyword color = blue!70>
                \Statep{\Call{Process}{$a$}}[process current data]
            \EndWhile
        \EndIf
    \end{algorithmic}
\end{tcblisting}
%! formatter = on

There is a lot of additional information about options and how they can be used. See discussion and full list in \cref{sec:customization-and-fine-tunning}.

\subsection{Statements}\label{sec:statements}

The macros \Macro{State} and \Macro{Statex} defined in \PackageName{algorithmicx} can still be used for single statements and have the same general behaviour.

For automatic handling of comments and multi-line text, the \Macro{Statep} macro is available, which should be used instead of \Macro{State}.

\begin{macro}{Statep}[\OArg{comment*}\MArg{text}\OArg{comment}\LArg{options}]
    The \Macro{Statep} macro corresponds to an statement that can extrapolate a single line. The continuation of each line is indented from the baseline and this indentation is based on the value indicated in the \OptionRef{statement indent} option.

    \MacroOptionsText
\end{macro}

As an example, observe lines~\ref{alg:lzw:output} and \ref{alg:lzw:add-to-table} of the LZW compression algorithm on \cpageref{alg:lzw:add-to-table}.

\subsection{Flow Control Blocks}\label{sec:flow-control-blocks}
Flow control is essentially based on conditionals and loop.

\subsubsection{The \Keyword{if} block}
This block is the standard \textit{if} block.

\begin{tcblisting}{}
    \begin{algorithmic}
        \State \Read $v$
        \If{$v < 0$}[is it negative?]
            \Statep{$v \gets -v$}[make it positive]
        \EndIf
    \end{algorithmic}
\end{tcblisting}

\begin{macro}{If}[\OArg{comment*}\MArg{text}\OArg{comment}\LArg{options}]
    \Macro{If} shows \Argument{text} (the condition) and must be closed with an \MacroRef{EndIf}, creating a block of nested commands.

    \BlockOptionsText
\end{macro}

\begin{macro}{EndIf}[\OArg{comment}\LArg{options}]
    \Macro{EndIf} closes its respective \MacroRef{If}.

    \MacroOptionsText
\end{macro}

\begin{macro}{Else}[\OArg{comment}\LArg{options}]
    This macro defines the \Keyword{else} part of the \MacroRef{If} statement.

    \BlockOptionsText
\end{macro}

\begin{macro}{Elsif}[\OArg{comment*}\MArg{text}\OArg{comment}\LArg{options}]
    \Macro{ElsIf} defines the \MacroRef{If} chaining. The argument \Argument{text} is the new condition.

    \BlockOptionsText
\end{macro}

\subsubsection{The \Keyword{switch} block}

%! formatter = off
\begin{tcblisting}{}
    \begin{algorithmic}
        \Statep{Get \Id{option}}
        \Switch{\Id{option}}
            \Case{1}[inserts new record]
                \Statep{\Call{Insert}{\Id{record}}}
            \EndCase
            \Case{2}[deletes a record]
                \Statep{\Call{Delete}{\Id{key}}}
            \EndCase
            \Otherwise
                \Statep{Print ``invalid option''}
            \EndOtherwise
        \EndSwitch
    \end{algorithmic}
\end{tcblisting}
%! formatter = on


\begin{macro}{Switch}[\OArg{comment*}\MArg{expression}\OArg{comment}\LArg{options}]
    The \Macro{Switch} is closed by a matching \MacroRef{EndSwitch}.

    \BlockOptionsText
\end{macro}

\begin{macro}{EndSwitch}[\OArg{comment}\LArg{options}]
    This macro closes a \MacroRef{Switch} block.

    \MacroOptionsText
\end{macro}

\begin{macro}{Case}[\OArg{comment*}\MArg{constant-list}\OArg{comment}\LArg{options}]
    When the result of the \Keyword{switch} expression matches one of the constants in \Argument{constants-list}, then the \Keyword{case} is executed. Usually the \Argument{constant-list} is a single constant, a comma-separated list of constants or some kind of range specification.

    \BlockOptionsText
\end{macro}

\begin{macro}{EndCase}[\OArg{comment}\LArg{options}]
    This macro closes a corresponding \MacroRef{Case} statement.

    \MacroOptionsText
\end{macro}

\begin{macro}{Otherwise}[\OArg{comment}\LArg{options}]
    A \Keyword{switch} structure can optionally use an \Keyword{otherwise} clause, which is executed when no previous \Keyword{case}s had a hit.

    \BlockOptionsText
\end{macro}

\begin{macro}{EndOtherwise}[\OArg{comment}\LArg{options}]
    This macro closes a corresponding \MacroRef{Otherwise} statement.

    \MacroOptionsText
\end{macro}

\subsubsection{The \Keyword{for} block}
The \Keyword{for} loop uses \Macro{For} and is also flavored with two variants: \Keyword{foreach} (\Macro{ForEach}) and \Keyword{forall} (\Macro{ForAll}).

%! parser = off
\begin{tcblisting}{}
    \begin{algorithmic}
        \For{$i \gets 0$ \To $n$}
            \Statep{Do something with $i$}
        \EndFor
        \ForAll{$\Id{item} \in C$}
            \Statep{Do something with \Id{item}}
        \EndFor
        \ForEach{\Id{item} in queue $Q$}
            \Statep{Do something with \Id{item}}
        \EndFor
    \end{algorithmic}
\end{tcblisting}
%! parser = on


\begin{macro}{For}[\OArg{comment*}\MArg{text}\OArg{comment}\LArg{options}]
    The \Argument{text} is used to establish the loop scope.

    \BlockOptionsText
\end{macro}

\begin{macro}{EndFor}[\OArg{comment}\LArg{option}]
    This macro closes a corresponding \MacroRef{For}, \MacroRef{ForEach} or \MacroRef{ForAll}.

    \MacroOptionsText
\end{macro}

\begin{macro}{ForEach}[\OArg{comment*}\MArg{text}\OArg{comment}\LArg{options}]
    Same as \MacroRef{For}.
\end{macro}

\begin{macro}{ForAll}[\OArg{comment*}\MArg{text}\OArg{comment}\LArg{options}]
    Same as \MacroRef{For}.
\end{macro}

\subsubsection{The \Keyword{while} block}
\Macro{While} is the loop with testing condition at the top.

\begin{tcblisting}{}
    \begin{algorithmic}
        \While{$n > 0$}
            \Statep{Do something}
            \Statep{$n \gets n - 1$}
        \EndWhile
    \end{algorithmic}
\end{tcblisting}

\begin{macro}{While}[\OArg{comment*}\MArg{text}\OArg{comment}\LArg{options}]
    In \Argument{text} is the boolean expression that, when \False, will end the loop.

    \BlockOptionsText
\end{macro}

\begin{macro}{EndWhile}[\OArg{comment}\LArg{options}]
    This macro closes a matching \MacroRef{While} block.

    \MacroOptionsText
\end{macro}

\subsubsection{The \Keyword{repeat}-\Keyword{until} block}
The loop with testing condition at the bottom is the \Macro{Repeat}/\Macro{Until} block.

\begin{tcblisting}{}
    \begin{algorithmic}
        \Repeat
            \Statep{Do something}
            \Statep{$n \gets n - 1$}
        \Until{$n \leq 0$}
    \end{algorithmic}
\end{tcblisting}

\begin{macro}{Repeat}[\OArg{comment}\LArg{options}]
    This macro starts the \Keyword{repeat} loop, which is closed with \MacroRef{Until}.

    \BlockOptionsText
\end{macro}

\begin{macro}{Until}[\OArg{comment*}\MArg{text}\OArg{comment}\LArg{options}]
    In \Argument{text} is the boolean expression that, when \MacroRef{True}, will end the loop.

    \MacroOptionsText
\end{macro}

\subsubsection{The \Keyword{loop} block}
A generic loop is build with \Macro{Loop}.

\begin{tcblisting}{}
    \begin{algorithmic}
        \Loop
            \Statep{Do something}
            \Statep{$n \gets n + 1$}
            \If{$n$ is multiple of 5}
                \Statep{\Continue}[restarts loop]
            \EndIf
            \Statep{Do something else}
            \If{$n \leq 0$}
                \Statep{\Break}[ends loop]
            \EndIf
            \Statep{Keep working}
        \EndLoop
    \end{algorithmic}
\end{tcblisting}

\begin{macro}{Loop}[\OArg{comment}\LArg{options}]
    The generic loop starts with \Macro{Loop} and ends with \MacroRef{EndLoop}. Usually the infinite loop is interrupted by and internal \MacroDef{Break} or restarted with \MacroDef{Continue}.

    \BlockOptionsText
\end{macro}

\begin{macro}{EndLoop}[\OArg{comment}\LArg{options}]
    \Macro{EndLoop} closes a matching \MacroRef{Loop} block.

    \MacroOptionsText
\end{macro}

\subsection{Constants and Identifiers}\label{sec:constants-and-identifiers}
A few macros for well known constants were defined: \MacroDef{True} (\True), \MacroDef{False} (\False), and \MacroDef{Nil} (\Nil).

The macro \Macro{Id} was created to handle ``program-like'' named identifiers, such as \Id{sum}, \Id{word\_counter} and so on.

\begin{macro}{Id}[\MArg{identifier}]
    \VCChange{
        type = updated,
        version = 0.91,
        description = {\Macro{Id} has been recoded to work in both text and math modes, preventing hyphenation},
    }[\Macro{Id}]
    Identifiers are emphasised: \latexinline!\Id{value}! is \Id{value}. Its designed to work in both text and math modes: \latexinline!$\Id{offer}_k$! is $\Id{offer}_k$.
\end{macro}

\subsection{Assignments and I/O}\label{sec:assignments-and-i/o}

To support teaching-like, basic pseudocode writing, the macros \MacroDef{Read} and \MacroDef{Write} are provided.

\begin{tcblisting}{}
    \begin{algorithmic}
        \Statep{\Read $v_1, v_2$}
        \Statep{$\Id{mean} \gets \dfrac{v_1 + v_2}{2}$}[calculate]
        \Statep{\Write \Id{mean}}
    \end{algorithmic}
\end{tcblisting}

The \Macro{Set} macro, although obsolete, can be used for assignments.

\begin{macro}{Set}[\MArg{lvalue}\MArg{expression} \Deprecated]
    \VCChange{
        type = new,
        version = 0.91,
    %! parser = off
        description = {\Macro{Set} can be used for assignments (deprecated in v0.99)},
        %! parser = on
    }[\Macro{Set}]%
    This macro expands to \MacroRef{Id}\latexinline!{#1} \gets #2!.

    this macro will no longer be supported. It will be kept as is for backward compatibility however.
    \VCChange{
        type = deprecated,
        version = 0.99,
        description = {\Macro{Set} is deprecated and will no longer be supported},
    }[\Macro{Set}]%
    As the handling of text and math modes should be done and its usage brings no evident advantage,

    \VCChange{
        type = removed,
        version = 0.99,
        description = {\Macro{Setl} has been removed from the package},
    }[\Macro{Setl}]%
    The macro \MacroRefInd{Setl} has been removed.
\end{macro}

\subsection{Procedures and Functions}\label{sec:procedures-and-functions}
Modularization uses \Macro{Procedure} or \Macro{Function}.

%! formatter = off
\begin{tcblisting}{}
    \begin{algorithmic}
        \Procedure{SaveNode}{\Id{node}}[saves a B\textsuperscript{+}-tree node to disk]
            \If{\Id{node}.\Id{is\_modified}}
                \If{$\Id{node}.\Id{address} = -1$}
                    \Statep{Set file writting position after file's last byte}[creates a new node on disk]
                \Else
                    \Statep{Set file writting position to \Id{node}.\Id{address}}[updates the node]
                \EndIf
                \Statep{Write \Id{node} to disk}
                \Statep{$\Id{node}.\Id{is\_modified} \gets \False$}
            \EndIf
        \EndProcedure
    \end{algorithmic}
\end{tcblisting}

\begin{tcblisting}{}
    \begin{algorithmic}
        \Function{Factorial}{$n$}[$n \geq 0$]
            \If{$n \in \{0, 1\}$}
                \Statep{\Return $1$}[base case]
            \Else
                \Statep{\Return $n \times \Call{Factorial}{n-1}$}[recursive case]
            \EndIf
        \EndFunction
    \end{algorithmic}
\end{tcblisting}
%! formatter = on

\begin{macro}{Procedure}[\MArg{name}\MArg{argument list}\OArg{comment}\LArg{options}]
    This macro creates a \Keyword{procedure} block that must be ended with \MacroRef{EndProcedure}.

    \BlockOptionsText
\end{macro}

\begin{macro}{EndProcedure}[\OArg{comment}\LArg{optons}]
    This macro closes the \MacroRef{Procedure} block.

    \MacroOptionsText
\end{macro}

\begin{macro}{Function}[\MArg{name}\MArg{argument list}\OArg{comment}\LArg{options}]
    This macro creates a \Keyword{function} block that must be ended with \MacroRef{EndFunction}. A \MacroDef{Return} is defined.

    \BlockOptionsText
\end{macro}

\begin{macro}{EndFunction}[\OArg{comment}\LArg{optons}]
    This macro closes the \MacroRef{Function} block.

    \MacroOptionsText
\end{macro}

For calling a procedure or function, \Macro{Call} should be used.

\begin{macro}{Call}[\MArg{name}\MArg{arguments}\LArg{options}]
    \label{call}
    \Macro{Call} is used to state a function or procedure call. The module's \Argument{name} and \Argument{arguments} are mandatory.

    \MacroOptionsText
\end{macro}

\subsection{Comments}\label{sec:comments}

The \Macro{Comment} macro defined by \PackageName{algorithmicx} has the same original behavior and has been redefined to handle styling options.

\begin{macro}{Comment}[\MArg{text}\LArg{options}]
    The redesigned version of \Macro{Comment} can be used with \MacroRef{State}, \MacroRef{Statex} and \MacroRef{Statep}. When used with \MacroRef{Statep}, it must be enclosed inside the text braces, but multi-line statements should work differently than expected.

    \MacroOptionsText
\end{macro}

%! formatter = off
\begin{tcblisting}{sidebyside = false}
    \begin{minipage}{7.5cm}
        \begin{algorithmic}<comment color = blue>%  for viewing purposes only
        \State Store the value zero in variable $x$\Comment{first assignment}
        \Statep{Store the value zero in variable $x$\Comment{first assignment}}
        \Statep{Store the value zero in variable $x$}[first assignment]% best choice
        \end{algorithmic}
    \end{minipage}
\end{tcblisting}
%! formatter = on

\begin{macro}{Commentl}[\MArg{text}\LArg{options}]
    While \MacroRef{Comment} pushes text to the end of the line, the macro \Macro{Commentl} is ``local''. In other words, it just puts a comment in place.

    Local comments follows regular text and no line changes are checked.

    \MacroOptionsText
\end{macro}

\begin{tcblisting}{}
    \begin{algorithmic}
        \If{$a > 0$~~\Commentl{special case}\\
        or\\
            $a < b$~~\Commentl{general case}\\}
            \Statep{Process data~~\Commentl{may take a while}}
        \EndIf
    \end{algorithmic}
\end{tcblisting}

\begin{macro}{CommentIn}[\MArg{text}\LArg{options}]
    \Macro{CommentIn} is an alternative to \textit{line comments} which usually extends to the end of the line. This macro defines a comment with a begin and an end. A comment starts with \CommentSymbol\ and ends with \CommentSymbolRight.

    \MacroOptionsText
\end{macro}


\begin{tcblisting}{}
    \begin{algorithmic}
        \If{$a > 0$ \CommentIn{special case} or $a < b$ \CommentIn{general case}}
            \Statep{Process data~~\Commentl{may take a while}}
        \EndIf
    \end{algorithmic}
\end{tcblisting}

\subsection{Documentation}\label{sec:documentation}
A series of macros are defined to provide the header documentation for a pseudocode.

\begin{tcblisting}{}
    \begin{algorithmic}
        \Description Calculation of the factorial of a natural number through successive multiplications
        \Require $n \in \mathbb{N}$
        \Ensure $f = n!$
    \end{algorithmic}
\end{tcblisting}

\begin{macro}{Description}
[~\Argument{description text}]
    The \Macro{Description} is intended to hold the general description of the pseudocode.
\end{macro}

\begin{macro}{Require}
[~\Argument{pre-conditions}]
    The required initial state that the code relies on. These are \textit{pre-conditions}.
\end{macro}

\begin{macro}{Ensure}
[~\Argument{post-conditions}]
    The final state produced by the code. These are \textit{post-conditions}.
\end{macro}

\begin{tcblisting}{}
    \begin{algorithmic}
        \Description Calculation of the factorial of a natural number through successive multiplications
        \Input $n$ (integer)
        \Output $n!$ (integer)
    \end{algorithmic}
\end{tcblisting}

\begin{macro}{Input}
[~\Argument{inputs}]
    This works as an alternative to \MacroRef{Require}, presenting \Keyword{input}.
\end{macro}
\begin{macro}{Output}
[~\Argument{outputs}]
    This works as an alternative to \MacroRef{Ensure}, presenting \Keyword{output}.
\end{macro}


\section{Customization and Fine Tunning}\label{sec:customization-and-fine-tunning}
As of version 0.99 of \PackageName{algxpar}, a series of options have been introduced to customize the presentation of algorithms. Colors and fonts that only apply to keywords, for example, can be specified, providing an easier and more convenient way to customize each algorithm.

The \MacroRef{AlgSet} macro serves this purpose.

\begin{macro}{AlgSet}[\MArg{options list}]
    This macro sets algorithmic settings as specified in the \Argument{options list}, which is key/value comma-separated list.

    All settings will be applied to the entire document, starting from the point of the macro call. The scope of a definition made with \MacroRef{AlgSet} can be restricted to a part of the document simply by including it in a \TeX\ group.
\end{macro}

\begingroup
\begin{tcblisting}{}
    \AlgSet{algorithmic indent = 1.5cm}
    \begin{algorithmic}
        \Statep{\Read $k$}
        \If{$k < 0$}
            \Statep{$k \gets -k$}
        \EndIf
        \Statep{\Write $k$}
    \end{algorithmic}
\end{tcblisting}
\endgroup%  ends the scope

If the settings are only applied to a single algorithm and not a group of algorithms in a text section, the easiest way is to include the options in the \texttt{algorithmicx} environment.

%! formatter = off
\begin{tcblisting}{}
    \begin{algorithmic}<keyword font = \sffamily\bfseries\itshape>
        \Statep{\Read $k$}
        \If{$k < 0$}
            \Statep{$k \gets -k$}
        \EndIf
        \Statep{\Write $k$}
    \end{algorithmic}
\end{tcblisting}
%! formatter = on

Named styles can also be defined using the \PackageName{pgfkeys} syntax.

%! formatter = off
\begin{tcblisting}{}
    \AlgSet{
        fancy/.style = {
            text color = green!40!black,
            keyword color = blue!75!black,
            comment color = brown!80!black,
            comment symbol = \texttt{//},
        }
    }
    \begin{algorithmic}<fancy>
        \Statep{\Commentl{Process $k$}}
        \Statep{\Read $k$}
        \If{$k < 0$}
            \Statep{$k \gets -k$}[back to positive]
        \EndIf
        \Statep{\Write $k$}
    \end{algorithmic}
\end{tcblisting}
%! formatter = on

Sometimes some settings need to be applied exclusively to one command, for example to highlight a segment of the algorithm.

%! formatter = off
\begin{tcblisting}{}
    \AlgSet{
        highlight/.style = {
            text color = red!60!black,
            keyword color = red!60!black,
        }
    }
    \begin{algorithmic}
        \Statep{\Commentl{Process $k$}}
        \Statep{\Read $k$}
        \If{$k < 0$}<highlight>
            \Statep{$k \gets -k$}[back to positive]
        \EndIf
        \Statep{\Write $k$}
    \end{algorithmic}
\end{tcblisting}
%! formatter = on

\subsection{Options}\label{sec:options}
\VCChange{
    type = new,
    version = 0.99,
    description = {Added support to style the main components of the pseudocode, such as keywords, comments and text and widths}
}[Options]%
This section presents the options that can be specified for the algorithms, either using \MacroRef{AlgSet} or the \Argument{options} parameter of the various macros.

\begin{option}{indent lines}{\Option{true} | \Option{false}}[false]
    \VCChange{
        type = new,
        version = 0.99.1,
        description = Added experimental support to show vertical indentantion lines,
    }[\Option{indent lines}]%
    With \Option{indent lines} vertical indentation lines are displayed in the pseudocode. This feature only works correctly if the start and end of the block are on the same page and at least two compilation steps are required for the lines to be positioned correctly.

    To change the style for the lines, use \OptionRef{line style}

    \textit{This feature is still experimental and incomplete.}
\end{option}

%! formatter = off
\begin{tcblisting}{}
    \begin{algorithmic}<indent lines>
        \For{$i \gets 0$ \To $N - 1$}
            \For{$j \gets$ \To $N - 1$}
                \If{$m_{ij} < 0$}
                    \Statep{$m_{ij} \gets 0$}
                \EndIf
            \EndFor
        \EndFor
    \end{algorithmic}

    \begin{algorithmic}<indent lines, noend>
        \For{$i \gets 0$ \To $N - 1$}
            \For{$j \gets$ \To $N - 1$}
                \If{$m_{ij} < 0$}
                    \Statep{$m_{ij} \gets 0$}
                \EndIf
            \EndFor
        \EndFor
    \end{algorithmic}
\end{tcblisting}
%! formatter = on

\begin{option}{line style}{\Argument{style}}[\{lightgray, thick\}]
    \VCChange{
        suppress,
        version = 0.99.1,
        description = SUPRESSED???,
    }[\Option{line style}]%
    Almost anything Ti\emph{k}Z can apply to a line can be used to set the indentation lines.

    To show indentation lines in algorithms, \OptionRef{indent lines} must be set to true.
\end{option}

%! formatter = off
\begin{tcblisting}{}
    \begin{algorithmic}<indent lines, line style = {orange, dotted, -latex, ultra thick}>
        \For{$i \gets 0$ \To $N - 1$}
            \For{$j \gets$ \To $N - 1$}
                \If{$m_{ij} < 0$}
                    \Statep{$m_{ij} \gets 0$}
                \EndIf
            \EndFor
        \EndFor
    \end{algorithmic}
\end{tcblisting}
%! formatter = on


\begin{option}{language}{\Argument{language}}[english]
    This key is used to choose the keyword language set for the current scope. The language keyword set should already have been loaded through the package options (see \cref{sec:package-usage-and-options}).
\end{option}

\begin{option}{noend}{}
    Structured algorithms use blocks for its structures, marking their begin and end. In pseudocode it is common to use a line to finish a block.
    Using the option \Option{end}, this line is suppressed.

    The result is similar to a program written in Python.
\end{option}


\begin{option}{end}{}
    This option reverses the behaviour of \OptionRef{noend}, and the closing line of a block presented.
\end{option}

%! formatter = off
\begin{tcblisting}{}
    \begin{algorithmic}<noend>
        \For{$i \gets 0$ \To $N - 1$}
            \For{$j \gets$ \To $N - 1$}
                \If{$m_{ij} < 0$}<end>
                    \Statep{$m_{ij} \gets 0$}
                \EndIf
            \EndFor
        \EndFor
    \end{algorithmic}
\end{tcblisting}
%! formatter = on

\begin{option}{keywords}{\Argument{list of keywords assignments}}
    This option allows to change a keyword (or define a new one). See \cref{sec:languages-and-translations} for more information on keywords and translations.
\end{option}

%! formatter = off
\begin{tcblisting}{}
    \begin{algorithmic}<
            keywords = {
                terminate = Terminate, %  new keyword
                then = \{, %  redefined
                endif = \}, %  redefined
                while = whilst,  %  redefined
            }
        >
        \While{\True}
            \If{$t < 0$}
                \Statep{Run the \Keyword{terminate} module}
            \EndIf
        \EndWhile
    \end{algorithmic}
\end{tcblisting}
%! formatter = on

\begin{option}{algorithmic indent}{\Argument{width}}[1em]
    The algorithmic indent is the amount of horizontal space used for indentation inner commands.

    This option actually sets the \PackageName{algorithmicx}'s \Macro{algorithmicindent}.
\end{option}

\begin{option}{comment symbol}{\Argument{symbol}}[\Macro{triangleright}]
    The default symbol that preceeds the text in comments is \Macro{triangleright} (\CommentSymbol), as used by \PackageName{algorithmicx}, and can be changed with this key.

    The current comment symbol is available with \MacroDef{CommentSymbol}. Do not change this symbol by redefining \Macro{CommentSymbol}, as font, shape and color settings will no longer be respected. Always use \Option{comment symbol}.
\end{option}

\begin{option}{comment symbol right}{\Argument{symbol}}[\Macro{triangleleft}]
    This is the symbol that closes a \MacroRef{CommentIn}. This symbol is set to \CommentSymbolRight\ and can be retrieved with the \MacroDef{CommentSymbolRight} macro. Do not attempt to change the symbol by redefining \Macro{CommentSymbolRight}, as font, shape and color settings will no longer be respected. Always use \Option{comment symbol right}.
\end{option}

\subsubsection{Fonts, shapes and sizes}
The options ins this section allows setting font family, shape, weight and size for several parts of an algorithm.

Notice that color are handled separately (see \cref{sec:colors}) and using \Macro{color} with font options will tend to break the document.

\begin{option}{text font}{\Argument{font, shape and size}}[\Empty]
    This setting corresponds to the font family, its shape and size and applies to the \Argument{text} field in each of the commands.
\end{option}

\begin{option}{comment font}{\Argument{font, shape and size}}[\Macro{slshape}]
    This setting corresponds to the font family, its shape and size and applies to all comments.
\end{option}

\begin{option}{keyword font}{\Argument{font, shape and size}}[\Macro{bfseries}]
    This setting sets the font family, shape, and size, and applies to all keywords, such as \Keyword{function} or \Keyword{end}.
\end{option}

\begin{option}{constant font}{\Argument{font, shape and size}}[\Macro{scshape}]
    This setting sets the font family, shape, and size, and applies to all constants, such as \Nil, \True\ and \False.

    This setting also applies when \MacroRef{Constant} is used.
\end{option}

\begin{option}{module font}{\Argument{font, shape and size}}[\Macro{scshape}]
    This setting sets the font family, shape, and size, and applies to both procedure and function identifiers, as well as their callings with \MacroRef{Call}.
\end{option}

\subsubsection{Colors}\label{sec:colors}
Colors are defined using the \PackageName{xcolors} package.

\begin{option}{text color}{\Argument{color}}
[.\mbox{\normalfont\normalcolor\normalsize~(dot)}]
    This setting corresponds to the color that applies to the \Argument{text} field in each of the commands.
\end{option}

\begin{option}{comment color}{\Argument{color}}[.!70]
    This setting corresponds to the color that applies to all comments.
\end{option}

\begin{option}{keyword color}{\Argument{color}}
[.\mbox{\normalfont\normalcolor\normalsize~(dot)}]
    This key is used to set the color for all keywords.
\end{option}

\begin{option}{constant color}{\Argument{color}}
[.\mbox{\normalfont\normalcolor\normalsize~(dot)}]
    This setting corresponds to the color that applies to the defined constant (see \cref{sec:constants-and-identifiers}) and also when macro \MacroRef{Constant} is used.
\end{option}

\begin{option}{module color}{\Argument{color}}
[.\mbox{\normalfont\normalcolor\normalsize~(dot)}]
    This color is applied to the identifier used in both \MacroRef{Procedure} and \MacroRef{Function} definitions, as well as module calls with \MacroRef{Call}. Notice that the arguments use \Option{text color}.
\end{option}

\subsubsection{Paragraphs}
Multi-line support are internally handled by \Macro{parbox}es.

\medskip
%! formatter = off
\begingroup
\begin{algorithmic}<show boxes>
    \Procedure{Euclid}{$a,b$}[The g.c.d. of a and b]
        \Statep{$r\gets a\bmod b$}
        \While{$r\not=0$}[We have the answer if r is 0]
            \Statep{$a\gets b$}
            \Statep{$b\gets r$}
            \Statep{$r\gets a\bmod b$}
        \EndWhile
        \Statep{\Keyword{return} $b$}[The g.c.d. is b]
    \EndProcedure
\end{algorithmic}
\endgroup
%! formatter = on

\medskip
The options in this section should be used to set how these paragraphs will be presented.

\begin{option}{text style}{\Argument{style}}[\Macro{RaggedRight}]
    This \Argument{style} is applied to the paragraph box that holds the \Argument{text} field in all commands.
\end{option}

\begin{option}{comment style}{\Argument{style}}[\Macro{RaggedRight}]
    This \Argument{style} is applied to the paragraph box that holds the \Argument{comment} field in all algorithmic commands. This setting will not be used with \MacroRef{Comment}, \MacroRef{Commentl} or \MacroRef{CommentIn}.
\end{option}

\begin{option}{comment separator width}{\Argument{width}}[1em]
    The minimum space between the text box and the \Macro{CommentSymbol}. This affects the available space in a line for keywords, text and comment.
\end{option}

\begin{option}{statement indent}{\Argument{width}}[1em]
    This is the \Macro{hangindent} set inside \MacroRef{Statep} statements.
\end{option}

\begin{option}{comment width}{\Option{auto}|\Option{nice}|\Argument{width}}[auto]
    There are two ways to balance the lengths of \Argument{text} and \Argument{comments} on a line, each providing different visual experiences.

    In automatic mode (\Option{auto}), the balance is chosen considering the widths that the actual text and comment have, trying to reduce the total number of lines, given there is not enough space in a single line for the keywords, text , comment and comment symbol. The consequence is that each line with a comment will have its own balance.

    The second mode, \Option{nice}, sets a fixed width for the entire algorithm, maintaining consistency across all comments. In that case, longer comments will tend to span a larger number of lines. The ``nice value'' is hardcoded and sets the comment width to \latexinline{0.4\linewidth}.

    Also, a fixed comment width can be specified.
\end{option}

% todo: insert the parameter indent when it starts working
\begin{comment}
    parameter indent = 0pt,
\end{comment}

\subsection{Languages and translations}\label{sec:languages-and-translations}
\VCChange{
    type = new,
    version = 0.99,
    description = {New mechanism to translation with separate files for each language},
}[Languages]%
A simple mechanism is employed to allow keywords to be translated into other languages.

%! formatter = off
\begingroup
\begin{tcblisting}{}
    \begin{algorithmic}<language = brazilian>
        \Procedure{Euclid}{$a,b$}
            \State $r\gets a\bmod b$
            \While{$r\not=0$}
                \State $a\gets b$
                \State $b\gets r$
                \State $r\gets a\bmod b$
            \EndWhile
            \Statep{\Keyword{return} $b$}
        \EndProcedure
    \end{algorithmic}
\end{tcblisting}
\endgroup
%! formatter = on


Creating a new keyword set uses the \Macro{AlgLanguageSet} macro.

\begin{macro}{AlgLanguageSet}[\MArg{language name}\MArg{keyword assignments}]
    This macro sets new values for known keywords as well as new ones. Once created, keywords cannot be deleted.

    In case a default keyword is not reset, the English version will be used.

    To create a new set, copy the file \texttt{algxpar-english.kw.tex} and edit it accordingly.

    Note that there is a set of keywords for the lines that close each block. These keys are provided to allow for more versatility in changing how these lines are presented. It is highly recommended that references to other keywords use the {Keyworkd} macro so that font, color and language changes can be made without any problems.

    In translations, these \textit{compound keywords} do not necessarily need to appear (see file \texttt{brazilian.kw.tex}, which follows the settings in \texttt{algxpar-english.kw.tex}). However, if defined, there will be different versions for each language.
\end{macro}

\begin{figure}
    \tcbinputlisting{listing file = algxpar-english.kw.tex, listing only}
\end{figure}

The mechanism behind \MacroRef{AlgLanguageSet} uses the \Macro{SetKeyword} macro, which is called to adjust the value of a single keyword\footnote{Macros like \Macro{algorithmicwhile} from the \PackageName{algorithimicx} package are no longer used.}. To retrieve the value of a given keyword, the \Macro{Keyword} macro must be used. It returns the formatted value according to the options currently in use for keywords.

\begin{macro}{SetKeyword}[\OArg{language}\MArg{keyword}\MArg{value}]
    The macro \MacroRef{SetKeyword} changes a given \Argument{keyword} to \Argument{value} if it exists; otherwise a new keyword is created.

    If \Argument{language} is omitted, the language currently in use is changed.

    See also the \OptionRef{keywords} option.
\end{macro}

\begin{macro}{Keyword}[\OArg{language}\MArg{keyword}]
    This macro expands to the value of a keyword in a \Argument{language} using the font, shape, size, and color determined for the keyword set.

    If \Argument{language} is not specified, the current language is used. \Argument{keyword} is any keyword defined for a language, including custom ones.
\end{macro}

\begin{tcblisting}{}
    \SetKeyword[german]{if}{wenn} % new
    Depending on the language, a keyword can take different forms: \Keyword{if} (English), \Keyword[german]{if} (German) or \Keyword[brazilian]{if} (Brazilian Portuguese).
\end{tcblisting}

\subsection{Other features}

\VCChange{
    version = 0.99,
    description = Added support to style named constants and function/procedures identifiers,
}[Features]%
These macros were added to support the styles defined for named constants and function and procedure identifiers.

\begin{macro}{Constant}[\OArg{name}]
    This macro presents \Argument{name} using font, shape, size and color defined for constants.
\end{macro}

\begin{macro}{Module}[\OArg{name}]
    This macro presents \Argument{name} using font, shape, size and color defined for procedures and functions.
\end{macro}


\section{``Forgotten'' features}

\VCChange{
    type = removed,
    version = 0.99,
    description = {Macros \Macro{TextString} and \Macro{VisibleSpace} accidentally removed},
}
Some features were forgotten from v0.91 to v.99, although it was not intentional. The macros \Macro{TextString} and \Macro{VisibleSpace} simply vanished into thin air.


\section{To do}
This is a \textit{todo} list:
\begin{itemize}
    \item Add font, shape, size and color settings to a whole algorithm;
    \item Add font, shape, size and color settings to line numbers;
    \item Add font, shape, size and color settings to identifiers.
\end{itemize}


\section{Examples}

\subsection{LZW revisited}
\begingroup
\begin{latexcode}
    \AlgSet{
        comment color = purple,
        comment width = nice,
        comment style = \raggedleft,
    }
\end{latexcode}

\AlgSet{
    comment color = purple,
    comment width = nice,
    comment style = \raggedleft,
}
\begin{algorithmic}[1]
    \Description LZW Compression using a table with all known sequences of bytes.
    \Input A flow of bytes
    \Output A flow of bits with the compressed representation of the input bytes
    \Statex
    \Statep{Initialize a table with all bytes}[each position of the table has a single byte]
    \Statep{Initilize \Id{sequence} with the first byte in the input flow}
    \While{there are bytes in the input}[wait until all bytes are processed]
        \Statep{Get a single byte from input and store it in \Id{byte}}
        \If{the concatention of \Id{sequence} and \Id{byte} is in the table}
            \Statep{Set \Id{sequence} to $\Id{sequence} + \Id{byte}$}[concatenate without producing any output]
        \Else
            \Statep{Output the code for \Id{sequence}}[i.e., the binary representation of its position in the table]\label{alg:lzw:output}
            \Statep{Add the concatention of \Id{sequence} and \Id{byte} to the table}[the table learns a longer sequence]\label{alg:lzw:add-to-table}
            \Statep{Set \Id{sequence} to \Id{byte}}[starts a new sequence with the remaining byte]
        \EndIf
    \EndWhile
    \Statep{Output the code for \Id{sequence}}[the remaining sequence of bits]
\end{algorithmic}
\endgroup

\subsection{LZW revisited again}
\begingroup
\begin{latexcode}
    \AlgSet{
        keyword font = \ttfamily,
        keyword color = green!40!black,
        text font = \itshape,
        comment font = \footnotesize,
        algorithmic indent = 1.5em,
        indent lines,
        noend,
    }
\end{latexcode}

\AlgSet{
    keyword font = \ttfamily,
    keyword color = green!40!black,
    text font = \itshape,
    comment font = \footnotesize,
    algorithmic indent = 1.5em,
    indent lines,
    noend,
}
\begin{algorithmic}[1]
    \Description LZW Compression using a table with all known sequences of bytes.
    \Input A flow of bytes
    \Output A flow of bits with the compressed representation of the input bytes
    \Statex
    \Statep{Initialize a table with all bytes}[each position of the table has a single byte]
    \Statep{Initilize \Id{sequence} with the first byte in the input flow}
    \While{there are bytes in the input}[wait until all bytes are processed]
        \Statep{Get a single byte from input and store it in \Id{byte}}
        \If{the concatention of \Id{sequence} and \Id{byte} is in the table}
            \Statep{Set \Id{sequence} to $\Id{sequence} + \Id{byte}$}[concatenate without producing any output]
        \Else
            \Statep{Output the code for \Id{sequence}}[i.e., the binary representation of its position in the table]\label{alg:lzw:output}
            \Statep{Add the concatention of \Id{sequence} and \Id{byte} to the table}[the table learns a longer sequence]\label{alg:lzw:add-to-table}
            \Statep{Set \Id{sequence} to \Id{byte}}[starts a new sequence with the remaining byte]
        \EndIf
    \EndWhile
    \Statep{Output the code for \Id{sequence}}[the remaining sequence of bits]
\end{algorithmic}
\endgroup


%% Index
\printindex

\end{document}
%%%%%%%%%%%%%%%%%%%%%%%%%%%%%%%%%%%%%%%%%%%%%%%%%%%%%%%%%%%%%%%%%%%%%%